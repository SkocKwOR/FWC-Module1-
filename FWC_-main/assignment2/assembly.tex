\def\mytitle{IMPLEMENTATION OF BOOLEAN LOGIC IN ASSEMBLY}
\def\myauthor{P.PAVAN KUMAR}
\def\contact{padmanabhunipavna0@gmail.com}
\def\mymodule{Future Wireless Communication (FWC)}
\documentclass[10pt, a4paper]{article}
\usepackage[a4paper,outer=1.5cm,inner=1.5cm,top=1.75cm,bottom=1.5cm]{geometry}
\twocolumn
\usepackage{graphicx}
\graphicspath{{./images/}}
\usepackage[colorlinks,linkcolor={black},citecolor={blue!80!black},urlcolor={blue!80!black}]{hyperref}
\usepackage[parfill]{parskip}
\usepackage{lmodern}
\usepackage{tikz}
%\documentclass[tikz, border=2mm]{standalone}
\usepackage{karnaugh-map}
%\documentclass{article}
\usepackage{tabularx}
\usepackage{circuitikz}
\usetikzlibrary{calc}


\renewcommand*\familydefault{\sfdefault}
\usepackage{watermark}
\usepackage{lipsum}
\usepackage{xcolor}
\usepackage{listings}
\usepackage{float}
\usepackage{titlesec}

\titlespacing{\subsection}{1pt}{\parskip}{3pt}
\titlespacing{\subsubsection}{0pt}{\parskip}{-\parskip}
\titlespacing{\paragraph}{0pt}{\parskip}{\parskip}
\newcommand{\figuremacro}[5]{
    \begin{figure}[#1]
        \centering
        \includegraphics[width=#5\columnwidth]{#2}
        \caption[#3]{\textbf{#3}#4}
        \label{fig:#2}
    \end{figure}
}

\lstset{
frame=single, 
breaklines=true,
columns=fullflexible
}

%\thiswatermark{\centering \put(0,-90){\includegraphics[scale=0.13]{IITH.jpeg}} }
\title{\mytitle}
\author{\myauthor\hspace{1em}\\\contact\\FWC22088\hspace{6.5em}IITH\hspace{0.5em}\mymodule\hspace{6em}ASSIGN-1}
\date{}
\begin{document}
	\maketitle
	\tableofcontents
	\begin{abstract}
	    To Obtain the Boolean Expression for the Logic circuit shown below
	  	\end{abstract}
	  	
	   \begin{circuitikz} \draw
(0,2) node[nand port] (mynand1) {}
(2,3) node[nand port] (mynand2) {}
(0,0) node[nand port] (mynand) {}
(2,-1) node[nand port] (mynand3) {}
(2,1) node[or port] (myor1) {}
(4,1) node[or port,number inputs =3] (myor2) {}
(mynand1.out) -- (myor1.in 1)
(mynand.out) -- (myor1.in 2)
(mynand2.out) -- (myor2.in 1)
(mynand3.out) -- (myor2.in 3)
(myor1.out) -- (myor2.in 2);
\node[left] at (mynand1.in 1) {\(A\)};
\node[left] at (mynand1.in 2) {\(B\)};
\node[left] at (mynand2.in 1) {\(A\)};
\node[left] at (mynand2.in 2) {\(A\)};
\node[left] at (mynand3.in 1) {\(C\)};
\node[left] at (mynand3.in 2) {\(C\)};
\node[left] at (mynand1.in 1)[ocirc] {};
\node[left] at (mynand.in 2) [ocirc] {};
\node[left] at (mynand.in 1) {\(B\)};
\node[left] at (mynand.in 2) {\(C\)};
\node[right] at (mynand1.out) {\((A'B')\)};
\node[right] at (mynand.out) {\((B'C')\)};
\node[right] at (mynand2.out) {\((A')\)};
\node[right] at (mynand3.out) {\((C')\)};
\node[right] at (myor2.out) {\(Y\)};
\end{circuitikz}
\begin{center}
Fig. 1(Y = A'B'+B'C'+A'+C')
\end{center}

	\section{Components}
  \begin{tabularx}{0.4\textwidth} { 
  | >{\centering\arraybackslash}X 
  | >{\centering\arraybackslash}X 
  | >{\centering\arraybackslash}X | }
\hline
 \textbf{Components}& \textbf{Values} & \textbf{Quantity}\\
\hline
Arduino & UNO & 1 \\  
\hline
JumperWires& M-M & 5 \\ 
\hline
Breadboard &  & 1 \\
\hline
\end{tabularx}
   \section{Implementation}
   	\subsection{METHOD-1}
   	\paragraph{}

The truth table  for Fig. 1 is available in Table-1
Using Boolean logic, output Y in Table 1 can be expressed in terms of the inputs A, B, C as Y= A'.B'+B'.C'+A'+C' ......(2.1)
D3,D4,D5 Pins of Arduino are configured as input pins instead of initializing A,B,C inside software,inputs are given manually as A,B,C.led will glow based on Y satisfying the Table-1
The code below realizes the Boolean logic for Y in Table-1

\begin{center}
\begin{lstlisting}
https://github.com/SkocKwOR/FWC_/blob/main/assignment2/ass1/ass.asm
\end{lstlisting}
	\begin{tabularx}{0.4\textwidth} { 
  | >{\centering\arraybackslash}X 
  | >{\centering\arraybackslash}X 
  | >{\centering\arraybackslash}X
  | >{\centering\arraybackslash}X | }
\hline
 A & B & C & Y \\
\hline
0 & 0 & 0 & 1 \\  
\hline
0 & 0 & 1 & 1 \\ 
\hline
0 & 1 & 0 & 1 \\
\hline
0 & 1 & 1 & 1 \\
\hline
1 & 0 & 0 & 1 \\  
\hline
1 & 0 & 1 & 0 \\ 
\hline
1 & 1 & 0 & 1 \\
\hline
1 & 1 & 1& 0 \\
\hline
\end{tabularx}
 \end{center}
\begin{center}
Table-1 
  \end{center}
\begin{center}
\subsection{METHOD-2}
     \begin{karnaugh-map}[4][2][1][$BC$][$A$]
        \minterms{0,1,2,3,4,6}
        \maxterms{5,7}
        \implicant{0}{2}
       \implicantedge{0}{4}{2}{6}
    \end{karnaugh-map}
\end{center}
\begin{center}
Fig. 2
\end{center}
    \paragraph{Karnugh Map :}
The expression in (2.1) can be minimized using the K-map in Fig 2. In Fig.2 ,the implicants in boxes 0,1,2,3 result iS A'
The implicants in boxes 0,4,2,6 result in C'
Thus, after minimization using Fig. 2, (2.1) can
be expressed as
Y=A'+ C'........(2.2).
Verify the truth table for Y in TABLE 1.
The code below realizes the Boolean logic for F in 2.2
\begin{lstlisting}
https://github.com/SkocKwOR/FWC_/blob/main/assignment2/ass2/ass2.asm
\end{lstlisting}
\bibliographystyle{ieeetr}
\end{document}